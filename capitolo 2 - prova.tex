\documentclass[11pt]{report}
% -------------------------------------------------------------------------------- %

% *** LANGUAGE ***
\usepackage[english,italian]{babel}
\usepackage[utf8]{inputenc}

% *** PAGE LAYOUT ***
\usepackage{geometry}     
\geometry{a4paper,tmargin=2cm,bmargin=2cm,lmargin=2cm,rmargin=2cm}
\usepackage{parskip}  
\usepackage{enumerate}  
\setlength{\parindent}{0pt}
\usepackage{subcaption}
\usepackage[font=small,labelfont=bf]{caption}


% *** DEFAULT FONT SELECTION ***
\usepackage[T1]{fontenc}
\usepackage{lmodern}
\renewcommand*\familydefault{\sfdefault} %	Only if the base font of the document is to be sans serif

% *** GRAPHICS *** 
\usepackage{graphicx}
\usepackage{color}
\usepackage{float}

% *** AMS MATH ***
\usepackage{amsmath, amsfonts, amssymb}

% *** ADDITIONAL MATH ***
\usepackage{empheq}

% *** TABLES ***
\usepackage{booktabs,topcapt}
\usepackage{multirow}

%*** BIBLIOGRAFIA ***
\usepackage{url}  % Per URL cliccabili
\usepackage[utf8]{inputenc}
\usepackage{csquotes}       % Consigliato con biblatex
\usepackage[backend=biber, style=numeric, url=true]{biblatex}
\addbibresource{bibliografia.bib}  % Nome file .bib



\begin{document}
	
	2.1 – Il veicolo prototipale
	→ Contesto del mezzo per il quale il software è pensato
	
	2.2 – Panoramica sui veicoli offroad
	→ Inquadramento delle esigenze dinamiche e ambientali
	
	2.3 – Il sistema attuale di bloccaggio
	→ Come funziona ora e quali limiti ha
	
	2.4 – Collocazione e ruolo del software di controllo
	→ Cosa deve fare il software all’interno del sistema veicolo
	
	2.5 – Ciclo di sviluppo software automotive
	→ In quale fase si inserisce il progetto, con quali strumenti
	
	2.6 – Architettura centraline
	→ Quali centraline sono coinvolte e come interagiscono
	
	2.7 –Architettura centraline
	→ Come le centraline si parlano, struttura del bus
	
	\chapter{Stato dell'arte}
	
	\section{Veicolo prototipale}

	L'architettura di controllo software per il bloccaggio del sistema differenziale oggetto di questo lavoro è stato progettato per essere, almeno in teoria, applicabile a una vasta gamma di veicoli a trazione integrale (4×4) destinati a operare in contesti fuoristrada. Tuttavia, per sviluppare, testare e validare concretamente la logica di controllo proposta, è stato necessario fare riferimento a un veicolo prototipale specifico, scelto per le sue caratteristiche costruttive e architettoniche allineate con gli obiettivi del progetto.
	
	Il veicolo scelto come riferimento è un mezzo commerciale leggero 4×4, con una struttura a telaio a longheroni, progettato per affrontare condizioni di lavoro impegnative, sia in ambito civile che militare. Le sue caratteristiche di modularità, robustezza e adattabilità lo rendono una piattaforma ideale per lo studio di  sottosistemi meccanici ed elettronici aggiuntivi.
	
	Dal punto di vista tecnico, questo veicolo pesa tra le 5 e le 7 tonnellate e presenta una trasmissione manuale a 6 marce, più la retromarcia. È dotato di un sistema di trazione 4×4 e sospensioni progettate per affrontare percorsi off-road. La sua velocità massima raggiunge i 130 km/h, e l'altezza da terra, insieme agli angoli di attacco e uscita, assicura una buona manovrabilità su terreni difficili. Il sistema frenante è a disco su entrambi gli assi, mentre l'impianto elettrico a 12 V alimenta le centraline e gli attuatori distribuiti lungo il veicolo.
	
	In particolare, la presenza di centraline elettroniche dedicate, una rete CAN per la comunicazione tra i vari sottosistemi, sensori di velocità delle ruote e attuatori già predisposti per altre funzioni veicolari, consente di simulare in modo realistico l’integrazione del software proposto. Questo approccio ci permette di sviluppare e validare il controllo in un contesto tecnico credibile, mantenendo al contempo la riservatezza sui dettagli specifici del veicolo reale.
	
	
	
	
	
	
	\section{Panoramica veicoli off-road }
	\section{Il sistema attuale di bloccaggio}
	\section{Collocazione e ruolo del software di controllo}
	\section{Architettura centraline}
	\section{Architettura centraline}
	

	
\end{document}