\documentclass[11pt]{report}
% -------------------------------------------------------------------------------- %

% *** LANGUAGE ***
\usepackage[english,italian]{babel}
\usepackage[utf8]{inputenc}

% *** PAGE LAYOUT ***
\usepackage{geometry}     
\geometry{a4paper,tmargin=2cm,bmargin=2cm,lmargin=2cm,rmargin=2cm}
\usepackage{parskip}  
\usepackage{enumerate}  
\setlength{\parindent}{0pt}
\usepackage{subcaption}
\usepackage[font=small,labelfont=bf]{caption}


% *** DEFAULT FONT SELECTION ***
\usepackage[T1]{fontenc}
\usepackage{lmodern}
\renewcommand*\familydefault{\sfdefault} %	Only if the base font of the document is to be sans serif

% *** GRAPHICS *** 
\usepackage{graphicx}
\usepackage{color}
\usepackage{float}

% *** AMS MATH ***
\usepackage{amsmath, amsfonts, amssymb}

% *** ADDITIONAL MATH ***
\usepackage{empheq}

% *** TABLES ***
\usepackage{booktabs,topcapt}
\usepackage{multirow}

%*** BIBLIOGRAFIA ***
\usepackage{url}  % Per URL cliccabili
\usepackage[utf8]{inputenc}
\usepackage{csquotes}       % Consigliato con biblatex
\usepackage[backend=biber, style=numeric, url=true]{biblatex}
\addbibresource{bibliografia.bib}  % Nome file .bib



\begin{document}
	
	2.1 – Il veicolo prototipale
	→ Contesto del mezzo per il quale il software è pensato
	
	2.2 – Panoramica sui veicoli offroad
	→ Inquadramento delle esigenze dinamiche e ambientali
	
	2.3 – Il sistema attuale di bloccaggio
	→ Come funziona ora e quali limiti ha
	
	2.4 – Collocazione e ruolo del software di controllo
	→ Cosa deve fare il software all’interno del sistema veicolo
	
	2.5 – Ciclo di sviluppo software automotive
	→ In quale fase si inserisce il progetto, con quali strumenti
	
	2.6 – Architettura centraline
	→ Quali centraline sono coinvolte e come interagiscono
	
	2.7 – Comunicazione CAN
	→ Come le centraline si parlano, struttura del bus
	
	\chapter{Stato dell'arte}
	
	\section{Veicolo prototipale}
	
	
	\section{Panoramica veicoli off-road }
	
	\section{Modellazione del veicolo e condizioni operative}
	
	Per l’implementazione del controllo, è fondamentale un modello dinamico semplificato del veicolo che consenta di stimare:
	
	\begin{itemize}
		\item la velocità angolare delle ruote;
		\item la coppia motrice;
		\item il coefficiente di aderenza ruota-suolo;
		\item eventuali differenze tra ruote (indicativo di slittamento).
	\end{itemize}
	
	Il sistema verrà modellato in \textbf{Simulink} per agevolare l’integrazione con tool di generazione automatica del codice (\textit{Embedded Coder}).
	
	\section{Strategie di controllo automatico}
	
	Una logica di controllo efficace per il bloccaggio automatico del differenziale dovrebbe tenere conto di condizioni specifiche, come:
	
	\begin{itemize}
		\item slittamento di una ruota rispetto all’altra sull’asse;
		\item velocità del veicolo;
		\item condizioni di trazione (derivate indirettamente);
		\item richieste del conducente (acceleratore, freno).
	\end{itemize}
	
	Una strategia semplice potrebbe essere definita in base al \textbf{differenziale di velocità} tra le ruote di uno stesso asse. Se questo valore supera una soglia per un tempo definito, si attiva il blocco:
	
	\begin{equation}
		\Delta \omega = |\omega_{sx} - \omega_{dx}| > \Delta \omega_{th}
	\end{equation}
	
	Dove:
	\begin{itemize}
		\item $\omega_{sx}$ e $\omega_{dx}$ sono le velocità angolari delle ruote sinistra e destra;
		\item $\Delta \omega_{th}$ è la soglia di attivazione;
	\end{itemize}
	
	A questa logica base, si possono aggiungere criteri di isteresi e tempo minimo di persistenza, per evitare attivazioni errate o instabili.
	
	\section{Architetture software per il controllo veicolo}
	
	Per l’implementazione, il controllo software sarà suddiviso in moduli:
	
	\begin{itemize}
		\item \textbf{Acquisizione sensori}: lettura segnali da ABS, encoder ruota, ecc.
		\item \textbf{Elaborazione segnali}: filtraggio, calcolo del delta velocità;
		\item \textbf{Logica decisionale}: attivazione/disattivazione blocco;
		\item \textbf{Uscita}: comando all’attuatore del blocco.
	\end{itemize}
	
	Il tutto sarà progettato in ambiente \textbf{Simulink}, con successiva generazione automatica del codice in C per implementazione sulla centralina ECU tramite toolchain MATLAB.
	
	
\end{document}