\chapter{Titolo del primo capitolo}

\section{Introduzione}

Primo capitolo della tesi. Esempio di citazione singola \cite{Bolognani-Zigliotto}.

Esempio di equazione c:

\begin{equation}
\label{EQN:esempio}
a + b = 42
\end{equation}

L'equazione si richiama nel testo cos�: (\ref{EQN:esempio}). Esempio di due equazioni incolonnate:

\begin{equation}
\label{EQN:esempio2}
\begin{array}{rcl}
a + b & = & 42 \\
d + e & = & f
\end{array}
\end{equation}

Attenzione alle righe vuote in \LaTeX\ anche dopo le formule e le figure!!!
Osservate le formule \ref{EQN:esempio},\ref{EQN:esempio2} riproposte di seguito con lo stesso testo:

Esempio di equazione:
\begin{equation}
\label{EQN:esempio3}
a + b = 42
\end{equation}
L'equazione si richiama nel testo cos�: \eqref{EQN:esempio3}. Esempio di due equazioni incolonnate:
\begin{equation}
\label{EQN:esempio4}
\begin{array}{rcl}
a + b & = & 42 \\
d + e & = & f
\end{array}
\end{equation}

Mentre tra la \eqref{EQN:esempio} e la \eqref{EQN:esempio2} il testo inizia un nuovo capoverso, tra la \eqref{EQN:esempio3} e la \eqref{EQN:esempio4} no. 

Altro esempio di formula, questa volta in linea: $4+m+c=44\,gatti$ in fila per $6$ col resto di $2$.

\bigskip

\lipsum[1-10]
