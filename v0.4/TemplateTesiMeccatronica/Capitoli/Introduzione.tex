\chapter{Introduzione}

L'introduzione costituisce la prima parte dell'elaborato ed estende quanto contenuto nel
sommario, orientando meglio la lettura. In essa vanno inserite le informazioni che stanno a
monte, logicamente e cronologicamente, al lavoro svolto nella tesi. Si compone
essenzialmente dei seguenti punti:
\begin{itemize}
\item spiegazione della natura del problema considerato
\item descrizione dei contenuti reperibili in letteratura relativamente al problema in
questione, corredata da esaurienti citazioni bibliografiche
\item scopo del lavoro
\item indicazione dei metodi di soluzione del problema
\item elenco schematico del contenuto dei vari capitoli.
\end{itemize}